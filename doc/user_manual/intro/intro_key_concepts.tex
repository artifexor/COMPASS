\section{Key Concepts}
\label{sec:key_concepts}

In this section, a few key concepts are introduced to give a somewhat deeper understanding of COMPASS and to allow the reader to understand some main design choices made by the author. This should also give indications about the strengths and draw-backs of the chosen approach.

\subsection*{Database Systems}
A database allows for storage, retrieval and filtering of the data of interest. While SQLite3 being stand-alone database also has some drawbacks, it was chosen for its performance and ease of use (compared to e.g. compared NoSQL databases).\\
SQLite3 encapsulates a database in a single file container, which is read from a storage medium (e.g. hard drive). Therefore, it does not require installation of a database service, which is one of the advantages of the current solution.

\subsection*{Configuration}
At startup, numerous configuration files are loaded, and at shutdown the current configuration state of COMPASS is saved.\\
The configuration is not just a matter of storing simple parameters of components, but also what components exist. To give an example: Each existing View is saved, and when the program is started again, the previously active Views are created.  The same holds for the filters, or the database interface/schema. \\
This way, a user can have a specific program configuration for a specific usage situation, which can be instantly reused for a different dataset, using have a completely configurable database schema or filter configuration. \\
This allows for a high degree of flexibility, but somewhat complicates software development.

\subsection*{Database Content \& Data Sources}

\begin{itemize}  
    \item Data Source: Single data source, e.g. a certain Radar
    \begin{itemize}  
        \item Identified by name, SAC/SIC, ...
    \end{itemize}
    \item Data Source Type (DSType)
    \begin{itemize}  
        \item Type of source technology, e.g. Radar, MLAT, ADS-B, ...
    \end{itemize}    
    \item Database Content (DBContent)
    \begin{itemize}  
        \item Type of data, e.g. ASTERIX Formats CAT048, CAT020, CAT021, ...
    \end{itemize}        
\end{itemize}
\ \\

A certain Database Content (\textbf{DBContent}) is defined by a name and has a collection of variables. For example, CAT048 and CAT062 are exist as DBContent, and each has variables holding time, position, Mode 3/A codes and Mode C heights and so on. From a database, if such a DBContent is present, it can be loaded and displayed.\\

To allow displaying data from different DBContent in the same system, so-called meta-variables were introduced, which hold variables that are present in some or all DBContent (with a possibly different name or unit).  For example, there meta-variable 'ToD', which is a collection of sub-variables for each existing DBContent and the respective 'Time of Day' variable. \\

A data source is defined by its name and SAC/SIC, and has a certain data source type (DSType). During an ASTERIX import, e.g. data source 'Wuerzburg' of DSType 'Radar' can be defined, and the respective DBContent CAT048 inserted into the database.

\subsection*{Data Source Lines}

For each data source, up to 4 different lines can be used (L1, L2, L3, L4). Such lines can either be used to distinguish data recorded from different network lines, but also to loaded different recording files for the same sensor. E.g. different tracker runs can be imported and analyzed by importing them into different data source lines.


\subsection*{Unique Target Numbers}
A \textbf{U}nique \textbf{T}arget \textbf{N}umber (UTN) groups together system track updates and/or sensor target reports.
This information can be created either by a general association task or created based on ARTAS TRI information (one UTN for each ARTAS track, with the associated sensor information).

\subsection*{ASTERIX Data Import}
If surveillance data is given in EUROCONTROL's ASTERIX format, it can be decoded using the jASTERIX library. This library allows adding new framings, categories and editions based on configuration only. The resulting JSON data is then mapped to DBContent variables stored in the datbase. The mapping between JSON keys and DBContent variables is configurable, allowing for a broad usage spectrum.

%\subsection*{JSON Data Import}
%If surveillance data is given in the JSON format, it can be mapped to DBObject variables and imported to a selected schema. Currently, only SCDB is supported for importing, but this will be extended in the future. The mapping between JSON keys and DBObject variables is configurable, allowing for a broad usage spectrum.

\subsection*{Data Loading}
In COMPASS, a unified data loading process was chosen, meaning that only exactly one common dataset is loaded, which can be inspected using Views. When started, data is incrementally read from the database, stored in the resulting dataset, and distributed to the active Views. Each time such a loading process is triggered, all Views clear their dataset and gradually update. \\
This makes working with the data somewhat easier to understand, since only one dataset exists, while on the other hand it does not allow several independent datasets (e.g. with different filters) to be loaded.


 
