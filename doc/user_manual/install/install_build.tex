\section{Building from Source Code}

Building from source is a somewhat difficult undertaking, and special care has to be taken to install the correct libraries. Therefore it is recommended to use the AppImage whenever possible. \\

If building from source is still a requirement, please contact the author for support using \href{mailto:compass@openats.at}{compass@openats.at}.

%TODO_V7 Comment on building from source referring to the git repository

% Currently, COMPASS can be installed as source code and compiled by the user. Any recent Linux installation with a C++11 capable compiler should work without issues. Regarding software version, no particularly new features were used, so older versions than the verified ones should also work. \\

% The following software has to be installed on the workstation:

% \begin{table}[H]
%   \center
%   \begin{tabular}{ | l | l | l |}
%     \hline
%     \textbf{Package} & \textbf{Description} & \textbf{Verified version} \\ \hline
%     g++ & C++11 capable C++ compiler & 9.0.0 \\ \hline
%     cmake & CMake build tool & 3.9.1 \\ \hline
%     qt5-default & Qt5 development package & 5.9.9 \\ \hline
%     libboost-dev & Boost development libraries & 1.62.0.1 \\ \hline
%     mysql++-dev & MySQL library bindings & 3.2.2 \\ \hline
%     libmysqlclient-dev & MySQL development library & 1.0.2 \\ \hline
%     libsqlite3-dev & Sqlite3 development files & 3.2.2 \\ \hline
%     libgdal-dev & Geospatial data abstraction library & 2.2.1 \\ \hline
%     log4cpp5-dev & Logging libary & 1.1.1 \\ \hline
%     libarchive-dev & Archive libary & 3.2.2 \\ \hline
%     libeigen3-dev & Linear algebra libary & 3.3.4 \\ \hline
%     libtbb-dev & Intel Thread Building Blocks & 2019~U8-1 \\ \hline
%     NemaTode & Cross platform C++ 11 NMEA Parser \& GPS Framework & \\ \hline
%     doxygen & Documentation generation & 1.8.13 \\ 
%     \hline
%   \end{tabular}
%   \caption{Required software/libraries}
% \end{table}

% To build the source code with the ASTERIX decoder jASTERIX, follow the instruction in \url{https://github.com/hpuhr/jASTERIX} (Installation with Building). \\

% To install the COMPASS source code, either download the latest released version from \\ \url{https://github.com/hpuhr/COMPASS/releases} \\
% or use the following command (git required) to clone the current repository:

% \begin{lstlisting}
% git clone https://github.com/hpuhr/COMPASS.git
% \end{lstlisting}

% Enter the COMPASS source folder, create a subfolder 'build', enter it and execute cmake to create a Makefile:

% \begin{lstlisting}
% cmake ..
% \end{lstlisting}

% The output should look like this:
% \begin{lstlisting}
% sk@golem:~/test/COMPASS$ cmake .
%   Path: /home/sk/test/COMPASS
% -- The C compiler identification is GNU 7.2.0
% -- The CXX compiler identification is GNU 7.2.0
% -- Check for working C compiler: /usr/bin/cc
% -- Check for working C compiler: /usr/bin/cc -- works
% -- Detecting C compiler ABI info
% -- Detecting C compiler ABI info - done
% -- Detecting C compile features
% -- Detecting C compile features - done
% -- Check for working CXX compiler: /usr/bin/c++
% -- Check for working CXX compiler: /usr/bin/c++ -- works
% -- Detecting CXX compiler ABI info
% -- Detecting CXX compiler ABI info - done
% -- Detecting CXX compile features
% -- Detecting CXX compile features - done
%   System: Linux-4.13.0-16-generic
%   Install Path: /home/sk/test/COMPASS/dist
%   Platform: Linux
% -- Looking for pthread.h
% -- Looking for pthread.h - found
% -- Looking for pthread_create
% -- Looking for pthread_create - not found
% -- Looking for pthread_create in pthreads
% -- Looking for pthread_create in pthreads - not found
% -- Looking for pthread_create in pthread
% -- Looking for pthread_create in pthread - found
% -- Found Threads: TRUE  
% -- Boost version: 1.62.0
% -- Found the following Boost libraries:
% --   regex
% --   system
% --   thread
% --   chrono
% --   date_time
% --   atomic
%   Boost_INCLUDE_DIR: /usr/include
%   Boost_LIBRARY_DIR: 
%   CMAKE_MODULE_PATH: /home/sk/test/COMPASS/cmake_modules
% ...
% -- Configuring done
% -- Generating done
% -- Build files have been written to: /home/sk/test/COMPASS
% \end{lstlisting}

% Then, compile the source by executing:

% \begin{lstlisting}
% make
% \end{lstlisting}

% The output should look like this:

% \begin{lstlisting}
% Scanning dependencies of target compass_autogen
% [  1%] Automatic MOC for target compass
% Generating MOC predefs moc_predefs.h
% Generating MOC source 2MJGWJB4P3/moc_mainwindow.cpp
% ...
% Generating MOC source 3JYSCEOBDA/moc_viewmanagerwidget.cpp
% Generating MOC compilation mocs_compilation.cpp
% [  1%] Built target compass_autogen
% Scanning dependencies of target compass
% [  2%] Building CXX object CMakeFiles/compass.dir/src/compass.cpp.o
% [  2%] Building CXX object CMakeFiles/compass.dir/src/buffer/arraylist.cpp.o
% ...
% [ 93%] Linking CXX shared library dist/lib/libcompass.so
% [ 93%] Built target compass
% Scanning dependencies of target compass_client_autogen
% [ 94%] Automatic MOC for target compass_client
% Generating MOC predefs moc_predefs.h
% Generating MOC source 2MJGWJB4P3/moc_mainwindow.cpp
% ...
% Generating MOC source 3JYSCEOBDA/moc_viewmanagerwidget.cpp
% Generating MOC compilation mocs_compilation.cpp
% [ 94%] Built target compass_client_autogen
% Scanning dependencies of target compass_client
% ...
% [100%] Linking CXX executable compass_client
% [100%] Built target compass_client
% \end{lstlisting}

% To install the COMPASS client, execute the following command in the build folder:

% \begin{lstlisting}
% sudo make install
% \end{lstlisting}

% This will install binary and a data folder to your system directories, commonly in '/usr/local/bin/', '/usr/local/lib/' and '/usr/local/compass/' (or similiar, depending on you Linux distribution.

% To start the COMPASS client run:
% \begin{lstlisting}
% compass_client
% \end{lstlisting}  
