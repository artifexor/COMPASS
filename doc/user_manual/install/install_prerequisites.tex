\section{Prerequisites}

\subsection{Operating System}

The binary will currently only be provided to (somewhat current) Linux 64-bit operating systems. For the following distributions, the AppImage was reported to be working correctly:

\begin{itemize}  
\item CentOS 7.3
\item Ubuntu 14.04, 16.04, 17.04, 18.04, 19.04, 20.04
\item Linux Mint 18.3
\end{itemize}
\ \\

For the following distributions, the AppImage was reported to have issues:

\begin{itemize}  
\item CentOS 6.*: Unfortunately the operating system's glibc versions are too old.
\end{itemize}
\ \\

If you have tried other operating systems as the ones listed here, it would be appreciated if you could provide feedback about your experience to \href{mailto:compass@openats.at}{compass@openats.at}.\\

Other operating systems (e.g. Windows or Mac) are currently not supported.

\subsection{Recommended Hardware}

The application will perform best on a workstation with at least the following minimum requirements:

\begin{itemize}  
\item CPU with at least 2 physical cores
\item Dedicated NVidia or ATI graphics card
\item 8GB of RAM or more
\end{itemize}
\ \\

Depending on the loaded datasize, more RAM or a better graphics card might be of advantage. \\

\includegraphics[width=0.5cm]{../../data/icons/hint.png} Please \textbf{note} that other graphics cards (Intel, Matrox) will in good probability also work, but are not supported. \\\\

An optimal setup could be similiar to:

\begin{itemize}  
\item Intel i5 or better (at least 2 physical cores)
\item Decent NVidia or ATI graphics card, e.g.
\begin{itemize}  
\item Nvidia Quattro, GT 740 or later, GTX 660 or later
\item Or equivalent ATI GPU models
\item Refer e.g. to \url{http://gpuboss.com/} for benchmarks
\end{itemize}
\item 16GB of RAM or more
\end{itemize}


\subsection{Graphics Cards \& Drivers}
\label{sec:graphics_installation}

Since there exist hundreds of different graphics card types, and numerous possible drivers for each of them, support will only be given for ATI or NVidia graphics cards, with their native drivers.

To find out what graphics cards are available in the system, use \textit{lspci} (might have to be installed):

\begin{lstlisting}
lspci | grep VGA
\end{lstlisting}

Output might look like this:

\begin{lstlisting}
01:00.0 VGA compatible controller: NVIDIA Corporation GP104 [GeForce GTX 1080]
\end{lstlisting}

To find out which graphics card and driver are being used, the program \textit{glxinfo} can be used (might have to be installed). Please execute the following command:

\begin{lstlisting}
glxinfo | grep OpenGL
\end{lstlisting}

The output might look something like this:

\begin{lstlisting}
OpenGL vendor string: NVIDIA Corporation
OpenGL renderer string: GeForce GTX 1080/PCIe/SSE2
OpenGL core profile version string: 4.5.0 NVIDIA 384.111
OpenGL core profile shading language version string: 4.50 NVIDIA
OpenGL core profile context flags: (none)
OpenGL core profile profile mask: core profile
OpenGL core profile extensions:
OpenGL version string: 4.5.0 NVIDIA 384.111
OpenGL shading language version string: 4.50 NVIDIA
OpenGL context flags: (none)
OpenGL profile mask: (none)
OpenGL extensions:
OpenGL ES profile version string: OpenGL ES 3.2 NVIDIA 384.111
OpenGL ES profile shading language version string: OpenGL ES GLSL ES 3.20
OpenGL ES profile extensions:
\end{lstlisting}

This reveals several important points:

\begin{itemize}  
\item OpenGL vendor string: This is the grapics card driver. Only NVidia or ATI ones are supported.
\item OpenGL version string: This is the OpenGl version, 3.0 or later is recommended.
\item OpenGL shading language version string: This is the OpenGL shader version, 3.0 or later is recommended.
\end{itemize}

\paragraph{Unsupported Graphics Cards or Drivers}

If different graphics cards or drivers are in use, the output of glxinfo might be similiar to this:

\begin{lstlisting}
OpenGL vendor string: nouveau
OpenGL renderer string: Gallium 0.4 on NVE6
OpenGL core profile version string: 3.1 (Core Profile) Mesa 9.2.5
OpenGL core profile shading language version string: 1.40
OpenGL core profile context flags: (none)
OpenGL core profile extensions:
OpenGL version string: 3.0 Mesa 9.2.5
OpenGL shading language version string: 1.30
OpenGL context flags: (none)
OpenGL extensions:
\end{lstlisting}

Or this:

\begin{lstlisting}
OpenGL vendor string: Intel Open Source Technology Center
OpenGL renderer string: Mesa DRI Intel(R) Ivybridge Desktop 
OpenGL core profile version string: 3.3 (Core Profile) Mesa 11.0.6
OpenGL core profile shading language version string: 3.30
OpenGL core profile context flags: (none)
OpenGL core profile profile mask: core profile
OpenGL core profile extensions:
OpenGL version string: 3.0 Mesa 11.0.6
OpenGL shading language version string: 1.30
OpenGL context flags: (none)
OpenGL extensions:
OpenGL ES profile version string: OpenGL ES 3.0 Mesa 11.0.6
OpenGL ES profile shading language version string: OpenGL ES GLSL ES 3.00
OpenGL ES profile extensions:
\end{lstlisting}

In such cases, the following issues can be expected and will currently not be addressed:

\begin{itemize} 
\item Application might not even start (OpenGL version error)
\item Slow display performance (OpenGL emulation by CPU-based Mesa driver)
\item Graphical display errors (wrong colors, artefacts, etc.) 
\end{itemize} 

Please know that they might also work, but no guarantees or support can be given at this moment. \\

If you encounter a white-only OSGView and shader errors are present in the console log, please refer to \nameref{ref:issue_shaders}.


