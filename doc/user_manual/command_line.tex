\chapter{Command Line Options}
\label{sec:command_line} 

Several command line options have been added to allow for semi-automated running of tasks or convinient usage. This also allows for (limited) automated batch processing of data. \\

\includegraphics[width=0.5cm]{../../data/icons/hint.png} Please note that configuration of the application still has to be performed using the GUI, therefore it is required to set up the application correctly before command line options can be used successfully.\\

\includegraphics[width=0.5cm]{../../data/icons/hint.png} Please also note that error or warning message (or related confirmations) will still halt the automatic running of tasks, to ensure that the user is always aware of occuring issues. \\

To get a list of available command line options use e.g.:
\begin{lstlisting}
./COMPASS-release_x86_64.AppImage --help
\end{lstlisting}

\begin{lstlisting}
Allowed options:
  --help                          produce help message
  -r [ --reset ]                  reset user configuration and data
  --create_new_sqlite3_db arg     creates and opens new SQLite3 database with 
                                  given filename, e.g. '/data/file1.db'
  --open_sqlite3_db arg           opens existing SQLite3 database with given 
                                  filename, e.g. '/data/file1.db'
  --import_view_points arg        imports view points JSON file with given 
                                  filename, e.g. '/data/file1.json'
  --import_asterix arg            imports ASTERIX file with given filename, 
                                  e.g. '/data/file1.ff'
  --asterix_framing arg           sets ASTERIX framing, e.g. 'none', 'ioss', 
                                  'ioss_seq', 'rff'
  --asterix_decoder_cfg arg       sets ASTERIX decoder config using JSON 
                                  string, e.g. ''{"10":{"edition":"0.31"}}'' 
                                  (including one pair of single quotes)
  --import_json arg               imports JSON file with given filename, e.g. 
                                  '/data/file1.json'
  --json_schema arg               JSON file import schema, e.g. 'jASTERIX', 
                                  'OpenSkyNetwork', 'ADSBExchange', 'SDDL'
  --import_gps_trail arg          imports gps trail NMEA with given filename, 
                                  e.g. '/data/file2.txt'
  --import_sectors_json arg       imports exported sectors JSON with given 
                                  filename, e.g. '/data/sectors.json'
  --auto_process                  start automatic processing of imported data
  --associate_data                associate target reports
  --start                         start after finishing previous steps
  --load_data                     load data after start
  --export_view_points_report arg export view points report after start with 
                                  given filename, e.g. '/data/db2/report.tex
  --evaluate                      run evaluation
  --export_eval_report arg        export evaluation report after start with 
                                  given filename, e.g. '/data/eval_db2/report.t
                                  ex
  --quit                          quit after finishing all previous steps
\end{lstlisting}
\ \\

If additional command line options are wanted please contact the author.

\section{Options}

\subsection{--create\_new\_sqlite3\_db filename}

Adds the supplied filename and creates a new SQLite3 database using the following tasks:

\begin{itemize}
\item \nameref{sec:task_open_database}
\begin{itemize}
 \item \nameref{sec:sqlite_open}
 \end{itemize}
 \end{itemize}
 
\subsection{--open\_new\_sqlite3\_db filename}

Adds the supplied filename and opens an existing SQLite3 database using the following tasks:

\begin{itemize}
\item \nameref{sec:task_open_database}
\begin{itemize}
 \item \nameref{sec:sqlite_open}
 \end{itemize}
 \end{itemize}

 \subsection{--import\_view\_points filename}

After a database was opened, adds the supplied filename and starts an import using the following tasks:

\begin{itemize}
 \item \nameref{sec:task_import_view_points}
\end{itemize}

 
\subsection{--import\_asterix filename}

After a database was opened, adds the supplied filename and starts an import using the following tasks:

\begin{itemize}
 \item \nameref{sec:task_import_asterix}
\end{itemize}


\subsection{--asterix\_framing framing}

When an Import ASTERIX Task is started the given framing is used, the following options exist:

\begin{itemize}
\item none:  Raw, netto, unframed ASTERIX data blocks, equivalent to the 'empty' value in the GUI
\item ioss:  IOSS Final Format
\item ioss\_seq: IOSS Final Format with sequence numbers
\item rff: Comsoft RFF format
\end{itemize}

\subsection{--asterix\_decoder\_cfg 'str'}

When an Import ASTERIX Task is started the given configuration is used, in which the editions and mapping can be specified for each category using a JSON string. \\

Using the following string the edition 0.31 can be set for category 010:  \\
'\{"10":\{"edition":"0.31"\}\}' (including one pair of single quotes) \\

In a nice formatting the string looks like this:
\begin{lstlisting}[basicstyle=\small\ttfamily]
'{
"10":
    {
        "edition":"0.31"
    }
}'
\end{lstlisting}
\ \\

Please note the string "10" to identify catgory 010. \\

For one or a number of categories, the following options can be set:

\begin{itemize}
\item "edition":  ASTERIX editions as string, e.g. "1.0"
\item "ref\_edition":  ASTERIX reserved expansion field as string, e.g. "1.9"
\item "spf\_edition": ASTERIX special purpose field as string, e.g. "ARTAS"
\item "mapping": Mapping used after decoding, e.g. "CAT010 to Radar"
\end{itemize}
\ \\

Please note that the naming must be exactly as in the GUI, otherwise the application quits with an error message.


\subsection{--import\_gps\_trail filename}

After a database was opened, adds the supplied filename and starts an import using the following tasks:

\begin{itemize}
 \item \nameref{sec:task_import_gps}
\end{itemize}

\subsection{--import\_sectors\_json}

After a database was opened, adds imports a previously exported sectors JSON file using the following tasks:

\begin{itemize}
 \item \nameref{sec:task_manage_sectors}
\end{itemize}


\subsection{--auto\_process}

After a data was imported, automatically starts the following tasks:

\begin{itemize}
 \item \nameref{sec:task_calc_radar_pos}* (If Radar data was imported)
 \item \nameref{sec:task_postprocess}
 \item \nameref{sec:task_associate_artas_tris}* (If ARTAS data with TRIs was imported)
\end{itemize}

\subsection{--associate\_data}

After a post-processed database exists, create target report associations using the following tasks:

\begin{itemize}
 \item \nameref{sec:task_associate_tr}
\end{itemize}

\subsection{--start}

After the other tasks were run, automatically starts the application into the management window.

\subsection{--load\_data}

After starting the application into the management window, a load process is triggered.


\subsection{--export\_view\_points\_report}

After starting the application into the management window, a export View Points as PDF process is triggered:

\begin{itemize}
 \item \nameref{sec:view_points_export_pdf}
\end{itemize}
\ \\

The given argument filename defines the report filename, with the report directory as the parent directory of the given filename.

\subsection{--evaluate}

After starting the application into the management window, a pre-configured evaluation run is triggered:

\begin{itemize}
 \item \nameref{sec:eval}
\end{itemize}
\ \\

\subsection{--export\_eval\_report}

After starting the application into the management window, a pre-configured evaluation run is triggered:

\begin{itemize}
 \item \nameref{sec:eval}
\end{itemize}
\ \\

The given argument filename defines the report filename, with the report directory as the parent directory of the given filename.

\subsection{--quit}

After the other tasks were run, automatically quits the application.

\section{Examples}

In this section, a few possible use cases are given.

\subsection{Create, Import ASTERIX \& Start}

This command creates a new database, imports ASTERIX data, processess the data and starts the management window.

\begin{lstlisting}
./COMPASS-release_x86_64.AppImage --create_new_sqlite3_db /data/test.db --import_asterix /data/test.ff --auto_process --start
\end{lstlisting}
\ \\

\subsection{Open \& Start}

This command opens an existing database and starts the management window.

\begin{lstlisting}
./COMPASS-release_x86_64.AppImage --create_new_sqlite3_db /data/test.db --import_asterix /data/test.ff --auto_process --start
\end{lstlisting}
\ \\

\subsection{Create, Import multiple ASTERIX \& Start}

This commands create a new database, imports multiple ASTERIX data files, processess the data and starts the management window.

\begin{lstlisting}
./COMPASS-release_x86_64.AppImage --create_new_sqlite3_db /data/test.db --import_asterix /data/test.ff --quit
./COMPASS-release_x86_64.AppImage --open_sqlite3_db /data/test.db --import_asterix /data/test2.ff --auto_process --start
\end{lstlisting}
\ \\

\subsection{Create, Import View Points \& Start}

This command creates a new database, imports ASTERIX data, processess the data and starts the management window.

\begin{lstlisting}
./COMPASS-release_x86_64.AppImage --create_new_sqlite3_db /data/test.db --import_view_points /data/view_points.json --auto_process --start
\end{lstlisting}
\ \\

\subsection{Create, Import View Points \& Start \& Export as PDF \& Quit}

This command creates a new database, imports ASTERIX data, processess the data, starts the management window and exports the View Points as PDF, and then quits.

\begin{lstlisting}
./COMPASS-release_x86_64.AppImage --create_new_sqlite3_db /data/test.db --import_view_points /data/view_points.json --auto_process --start --export_view_points_report /data/test.db/report.tex --quit
\end{lstlisting}
\ \\
